\chapter{Kurzfassung}

\begin{german}
In dieser Arbeit geht es um die Erkennung von unerreichbarem Code im Quellcode. 
Unerreichbarer Code hat zwei verschiedene Ursachen, zum einen durch unbedingte Sprünge und zum anderen durch undurchführbare Bedingungen.

Unerreichbarer Code wird nicht nur durch spezielle statische Code-Analyse-Tools überprüft, sondern auch durch den Compiler oder eine integrierte Entwicklungsumgebung als Kontrolle bereitgestellt. Es gibt viele Lösungen für verschiedene Programmiersprachen, die unerreichbaren Code jedoch nicht in gleichem Maße prüfen. So stellen Compiler in der Regel unerreichbaren Code nicht fest, andere sind so konzipiert, dass sie einen Übersetzungsfehler liefern, wenn unerreichbarer Code auftritt.


In dieser Arbeit wurde das Erkennen von unerreichbarem Code in eine bestehende statische Codeanalyse eingebaut, die auf Structured Text abzielt, eine Pascal-ähnliche Sprache, die in der Norm IEC-61131-3 enthalten ist. Diese Analyse erfordert die Darstellung des zu analysierenden Programms in Form eines Kontrollflussgraphen, denn durch diese grafische Darstellung werden Sequenzen ohne Sprünge in Knoten übertragen, während Sprünge in Form von Kanten dargestellt werden. Mögliche Werte werden für jede Variable in Form von Einschränkungen mitgetragen. Diese Einschränkungen können in eine Sprache übersetzt werden, die ein SMT-Solver auf Erfüllbarkeit prüft. 

Zu diesem Zeitpunkt funktioniert der implementierte Ansatz nur auf kleinen, konstruierten Beispielen. Dafür ist die implementierung in der Lage besondere Fälle von unreachable code zu finden, die kein anderes untersuchtes Tool in der Lage war zu finden. Da nicht alle Sprachfeatures berücksichtigt wurden (z.b. Arrays, Strukturen) ist die Implementierung nicht komplett und sollte daher eher als unvollständig betrachtet werden.
\end{german}