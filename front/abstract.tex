\chapter{Abstract}
\label{cha:abstract}


This thesis is about the detection of unreachable code in source code. %This type of error occurs in two forms, either due to unconditional jumps using imperative statements, which is easily detected, or due to infeasible conditions, which is harder to detect, since a broader context is needed to determine feasibility.
Unreachable code has two different causes, on the one hand it is caused by unconditional jumps by imperative statements, which is easily detected, on the other hand due to infeasible conditions, since a broader context is needed to determine feasibility.

Unreachable code is not only checked by specific static code analysis tools,  but also provided as a check by the compiler or an integrated development environment. Many solutions for different programming languages exist, but do not check unreachable code to the same extent. For example, compilers usually do not even determine unreachable code, others are designed to not even compile when unreachable code is encountered (e.g., javac).


In this thesis this check was introduced to an existing static code analysis, which targets structured text, a pascal-like language included in the IEC-61131-3 standard. This analysis requires to represent the program in form of a control flow graph and simplifies passing every path. The context contains possible values for every variable in form of predicates. These predicates may be checked for feasibility by an SMT-solver, which is able to determine possible concrete values for each variable. 

At this point, the implemented approach only works on small, constructed examples. However, the implementation is able to find special cases of unreachable code that no other tool examined was able to find. Since not all language features were taken into account (e.g. arrays, structures) the implementation is not complete and should therefore rather be considered as incomplete.

% Funktioniert auf kleinen Beispielen, manche Sprach features nicht berücksichtigt (Strukturen, Arrays, etc.)
% Funktioniert auf nicht zu komplexen Beispielen (Loops bringen probleme mit sich --> Laufzeit)
% Eher als Prototyp zu betrachten, wird aber in Zukunft weiter geführt werden.