%!TEX root = ../main.tex

\chapter{State of the Art}
\label{cha:state of the art}

Analyzing code is a central part of software development. Naturally this process can be automated to some degree using various static code analysis tools. 
Some of these tools are applied at different stages during development ranging from compilation time to static code analysis in the cloud. 
Generally static code analysis benefits by detecting errors and problematic constructs early, but also highlights code that does not adhere to coding guidelines.


In this chapter multiple static code analysis tools will be presented, ranging in quality for detecting bugs (with a focus on unreachable code).
The accuracy for finding unreachable code increases with each tool presented. 
The examples are provided in form of Java source code beginning with the Java compiler (OpenJDK \cite{OpenJDK} implementation) itself, which verifies the correctness of source code before code generation. 


One of the most important tools a programmer uses is arguably the Integrated Development Environment (IDE), which is capable of additional code inspection. 
The most popular Java IDEs have been Eclipse \cite{incCommunityOpenInnovation} and IntelliJ \cite{IntelliJIDEACapable}. 
Possibly both may be improved by installing additional plugins for checking code.


Sonarqube \cite{sonarqube} is a static code analysis tool hosted in the cloud and usually is part of a continous integration pipeline. It offers a wide variety of checks including a more in depth unreachable code analysis.


Some instances of unreachable code may not be found using the tools mentioned before. For a more advanced approach statements have to be evaluated. Infeasible path conditions, described in Section \ref{sub:infeasible code}, may be found by using a mathematical prover or SMT-Solver.
\clearpage
\pagebreak
\section{Java Compiler}
\label{sec:Java compiler}
% Javac also used by IDEs. 

The Java compiler (OpenJDK \cite{OpenJDK}) performs checks regarding unreachable code that strictly results in a compile time error. 
In the official documentation \cite{Chapter14Blocks} unreachable code is defined as follows:
\begin{quote}
This section is devoted to a precise explanation of the word "reachable." 
The idea is that there must be some possible execution path from the beginning of the constructor, method, instance initializer, or static initializer that contains the statement to the statement itself. The analysis takes into account the structure of statements. Except for the special treatment of while, do, and for statements whose condition expression has the constant value true, the values of expressions are not taken into account in the flow analysis.
\label{quote:Java unreachable definition}
\end{quote}

Since values are not considered during this analysis, it is impossible to detect infeasible code as described in Listing \ref{code:Java infeasible undetected}.
 Loops are the exception here, but only conditions containing literal boolean values will be checked and may result in a compile time error as demonstrated by Listing \ref{code:Java loop unreachable}. The rationale behind not considering the values of variables and even constants is to use them as flags (see Listing \ref{code:Java flags}).


Unexpected return statements turn the following statements unreachable. As shown in Listing \ref{code:Java unexpected return} and mentioned before, conditions of if-then-else blocks are not evaluated, even if they contain only (boolean) literals. 


Unexpected break statements turn the following statements within the current scope of the loop unreachable as shown in Listing \ref{code:Java unexpected break}.


\begin{program}[h!]
	\begin{JavaCode}
void testUnreachableWhile() {
	int x = 3;
	while(false || false && true) x = 10; // Compile time error
}

void testUnreachableWhileNoError() {
	int x = 3;
	while(false || false && true || x == 4) x = 10; // No error
}

void testUnreachableWhileBoolean() {
	int x = 3;
	boolean y = false;
	while(false || false && true || y) x = 10; // No error
}\end{JavaCode}
	\caption{The Java compiler evaluates conditions of loops, if, and only if, they contain literal boolean values only.}
	\label{code:Java loop unreachable}
\end{program}

\begin{program}[h!]
	\begin{JavaCode}
void testSimpleIf() {
	int x = 3;
	if(false) x = 10; // No error
}\end{JavaCode}
	\caption{Simple if-then-else statements do not evaluate the condition at all.}
	\label{code:Java infeasible undetected}
\end{program}

\begin{program}[h!]
	\begin{JavaCode}
void testFlag() {
	int x = 3;
	final boolean DEBUG = false;
	if(DEBUG) x = 10; // No error
}\end{JavaCode}
	\caption{The rationale behind not even considering constant values is the usage of flags.}
	\label{code:Java flags}
\end{program}

\begin{program}[h!]
	\begin{JavaCode}
void testSystemExit() {
	System.exit(0);
	int x = 0; // no error - System.exit is not handled like a return statement
}

void testAlwaysReturn() {
	if(true) return;
	int x = 0; // no error - condition is not checked
}

void testReturn() {
	return;
	int i = 3; // Compilation error - unreachable statement
}

void testReturnInAllBranches() {
	boolean x = true;
	if(x) return;
	else return;
	x = false; // Compilation error - unreachable statement
}

void testReturnInWhile() {
	while(true) {
		return;
	}
	int i = 34; // Compilation error - unreachable statement
}\end{JavaCode}
	\caption{Examples of unreachable code due to unexpected return statements. Interestingly System.exit(), a statement that does terminate the program, is not handled like a return statement.}
	\label{code:Java unexpected return}
\end{program}

\begin{program}[h!]
	\begin{JavaCode}
void testBreak() {
	while(true) {
		int i = 3;
		break; 
		i++; // Compilation error - unreachable statement
	}
}

void testBreak2() {
	while(true) {
		int i = 3;
		if(true) break; 
		i++; // no error - condition is not checked
	}
}\end{JavaCode}
	\caption{Examples of unreachable code due to unexpected break statements.}
	\label{code:Java unexpected break}
\end{program}
%% TODO Liveness beschreiben
% Regarding unreachable code the Java compiler performs a data-flow analysis. The liveness of each statement will be analyzed. The implemented AST-model makes use of the visitor pattern \cite{gammaDesignPatternsElements}. 


\clearpage
\pagebreak
\section{Integrated Development Environments}
\label{sec:intelliJ}
Integrated Development Environments (IDE) are used by almost every software developer daily and are an integral tool for creating and refactoring code.
In comparison to simple text editors IDEs only support a few programming languages by providing some sort of static code analysis to assist a developer while writing code. 
Depending on the IDE, static code analysis can be configured to be more or less strict. 
When the IDE contains an extensive plugin system, then custom-made analysis tools may be used for even more precise method to find potential bugs.


Regarding unreachable code the following IDEs were analyzed:
\begin{itemize}
	\item \emph{Eclipse} \cite{incCommunityOpenInnovation} is an IDE primarily focused on developing Java projects. The IDE provides its own compiler \cite{teamJDTCoreComponent}, which does not only compile projects, but is also responsible for static code analysis. Additional to the checks by the OpenJDK \cite{OpenJDK} Java compiler, conditions in if-then-else expressions are also evaluated and checked by default. When values of variables are known, then they will be considered in the analysis as shown in Listing \ref{code:Java eclipse intelij always true or false}. Generally, the eclipse Java compiler is very basic regarding unreachable code detection. As shown in Listing \ref{code:Java eclipse fail to find this} its compiler does not even consider the same condition in an if/else-if block. Other datatypes (e.g., int) are also taken into account. Besides that Eclipse is extensible and many additional tools for more sophisticated analysis exist.
	
	% A basic, configurable inspection tool is integrated. Additional to the checks by the Java compiler, conditions in if-then-else expressions are also evaluated and checked by default. When values of variables are known, then they will be considered in the analysis as shown in Listing \ref{code:Java eclipse intelij always true or false}. Generally, the inspector is very basic regarding unreachable code detection. As shown in Listing \ref{code:Java eclipse fail to find this} its inspector does not even consider the same condition in an if/else-if block. Other datatypes (e.g., int) are also taken into account. Besides that Eclipse is extensible and many additional tools for more sophisticated analysis exist.
	\item \emph{IntelliJ Idea} \cite{IntelliJIDEACapable} is an IDE used for developing projects in various popular programming languages (e.g., Java, C\#, Python, JavaScript). For the Java language, a code inspector can be configured to find different categories of errors. By default two important rules are activated to deliver warnings: 
	\begin{itemize}
		\item \emph{Constant conditional expressions}, which do not only check for boolean literals in conditions, but also take values of variables, in case they always have the same value, and even other data types (e.g., int) into account as shown in Listing \ref{code:Java eclipse intelij always true or false}
		\item \emph{Duplicate condition} that have already been checked. Even though this may sometimes be intended, in most cases they are overlooked or may even not be reached, when the method terminates in the branch followed by the former condition as shown in Listing \ref{code:Java intelij duplicate conditions}. 
	\end{itemize}
	IntelliJ also provides a plugin system and therefore the possibility integrating an even better analysis is given.
\end{itemize}

% Beispiele für eclipse + IntelliJ

\begin{program}[h!]
	\begin{JavaCode}
void testAlwaysTrueCondition() {
	if(true) return; // Warning: Condition is always true
	int i = 3;
}

void testAlwaysFalse() {
	boolean x = true;
	boolean y = false;
	if(x && y) int i = 3; // Warning: Condition is always false
}

void testUnreachableWhile() {
	int x = 3;
	while(x > 4) x = 10; // Warning: Condition is always false
}\end{JavaCode}
	\caption{Internal code inspection tools of the IDEs \emph{Eclipse} and \emph{IntelliJ Idea} are capable of detecting conditions, which always evaluate to true or false.}
	\label{code:Java eclipse intelij always true or false}
\end{program}

\begin{program}[h!]
	\begin{JavaCode}
void testCondAlreadyChecked(boolean x) {
	int y = 0;
	if(x) {
		y = 3;
	} else if(x) { // Condition already checked before
		y = 4;
	}
	y++;
}\end{JavaCode}
\caption{Eclipse does not even check for duplicate conditions within the same if/else-if block. In contrast IntelliJ does find this issue.}
\label{code:Java eclipse fail to find this}
\end{program}

\begin{program}[h!]
	\begin{JavaCode}
void duplicateConditions(int s_operationHour, int m_operationHour) {
	if(s_operationHour != m_operationHour) { // Warning: Duplicate condition
		m_operationHour = s_operationHour;
		if (s_operationHour == 0) {
			foo();        
		}
		return;
	}
	if(s_operationHour == 0) {
		m_operationHour = 0;
	} else {
		// Unreachable, since the condition was checked already and returned
		if(s_operationHour != m_operationHour) { // Warning: Duplicate condition
			s_operationHour = m_operationHour;
		}
	}
}\end{JavaCode}
	\caption{Even though these warnings do not hint to the unreachable code directly. Duplicate condition in different branches of an if statement show a code smell prone to errors. In this particular case the else branch of the second if-then-else block cannot be reached, since the method already returned. Removing the return statement in line 7 removes this type of problem, but the second condition would always evaluate to false. }
	\label{code:Java intelij duplicate conditions}
\end{program}
\newpage
\section{Sonarqube and Sonarlint} % Could be interessting since it is very strict and performs more advanced analysis
\label{sec:sonar}
\emph{Sonarqube} \cite{sonarqube} and \emph{Sonarlint} \cite{SonarLintFixIssues} are products created by \emph{Sonarsource} \cite{CodeQualityCode}. Sonarqube is a static code analysis solution used in conjunction with a build server.
Commonly this analysis is either triggered after pushing commits, scheduled at certain times of the day or a combination of both.
Sonarqube supports a multitude of programming languages like Java, C\#, Python, COBOL and more. Another language may be supported in form of a plugin.


Sonarlint is available for some IDEs (including Eclipse and IntelliJ) in form of a plugin to provide immediate feedback. Note that this plugin only covers a subset of functionalities of sonarqube, but still provides better support than many internal inspectors built into IDEs.

\newpage
% Mutations beschreiben
Using the same example as before (seen in Listing \ref{code:Java intelij duplicate conditions}) Sonarqube detects that this expression always evaluates to false, which provides better information than the internal code inspector of IntelliJ, as demonstrated in Listing \ref{code:Java sonarqube duplicate conditions}. The example shown in Listing \ref{code:Java sonarqube hard example} cannot be detected. It appears that reassignments in loops are not evaluated and therefore no error is delivered. This type of error may only be detected by evaluating each statement.


\begin{program}[h!]
	\begin{JavaCode}
void duplicateConditions(int s_operationHour, int m_operationHour) {
	if(s_operationHour != m_operationHour) {
		m_operationHour = s_operationHour;
		if (s_operationHour == 0) {
			foo();        
		}
		return;
	}
	if(s_operationHour == 0) {
		m_operationHour = 0;
	} else {
		if(s_operationHour != m_operationHour) { // SonarLint: Change this Condition so it does not always evaluate false
			s_operationHour = m_operationHour;
		}
	}
}\end{JavaCode}
	\caption{The same example as seen in Listing \ref{code:Java intelij duplicate conditions}, but sonarqube reports the problem correctly. }
	\label{code:Java sonarqube duplicate conditions}
\end{program}

\begin{program}[h!]
	\begin{JavaCode}
void testLoop(boolean pred) {
	int x = 1;
	do {
		boolean b = x != 1;
		// b can never be true, this will possibly result in an infinite loop.
		if(b) {
			pred = false;
		}
	} while(pred);
}

void testDeadCode() {
	int i, j;
	i = 3;
	for(j = 1; j < i; ++j) {
		// ...
	}
	// i must be equal to j
	if(i != j) { 
		// ...
	}
}\end{JavaCode}
	\caption{Neither the internal code inspector of IntelliJ nor Eclipse nor sonarlint/-qube were able to catch these errors. }
	\label{code:Java sonarqube hard example}
\end{program}

\clearpage
\section{Joogie}
\label{sec:sca paper}
Joogie \cite{arltJoogieInfeasibleCode2012, arltJoogieJavaJimple2013} identifies infeasible code of any Java-project using a verification system. 
At first the source code will be transformed into a 3-address intermediate representation of Java bytecode using the Soot \cite{Soot} framework. 

Afterwards the Java bytecode is transformed into the Boogie language \cite{BoogieIntermediateVerification}, which is an imperative intermediate verification language for analyzing high-level programming languages.
During the translation the memory model has to be preserved. The heap is represented by a two-dimensional array. The first index refers to the object's address and the second to the field of the object.

Loops are unwound into three unwindings: the first and last iteration, as well as an abstract unwinding of the rest.
The result is a program containing no loops. Every assignment inside the loop replaces the value into non-determined values. Therefore loops are not fully evaluated.


Methods are checked in an intraprocedural manner. Therefore variables that may change become non-deterministic. 


Afterwards the Boogie program will be called by the verifier. 
The foundation of the verifier is the SMT-solver Z3.



When infeasible code is encountered, the java bytecode will be transformed back into Java and a violation will be reported.


This tool was tested on 4 bigger open source software projects including ArgoUML \cite{ArgoUMLResourcesSite}, Rachota \cite{RachotaTimetracker} and Freemind \cite{MainPageFreeMind}, and only few false warnings were reported.
The time needed for analysis was tolerable, since usually only small code fragments have to be analyzed instead of the whole project.

Since this is an older project relying on a special version of Z3 \cite{demouraZ3EfficientSMT2008}, which is not available anymore, it could not be tested in the course of this thesis. Neither of these two research papers \cite{arltJoogieInfeasibleCode2012, arltJoogieJavaJimple2013} do provide concrete examples, which instances of unreachable and infeasible code can be found.

This tool is probably as capable as Sonarqube \cite{sonarqube} in finding unreachable code.

