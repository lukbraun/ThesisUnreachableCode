%!TEX root = ../main.tex

\chapter {Theoretical Foundation \textasciitilde 15 sites}
\label {cha:theoretical foundation}

\section{Unreachable Code  \textasciitilde 1 site}
\subsection{Difference to Dead Code \textasciitilde 1/3 site}
\subsection{Difference to Unused Code \textasciitilde 1/3 site}
\subsection{Difference to Unnecessary Code \textasciitilde 1/3 site}
\section{Kemro-IEC 61131-3 \textasciitilde 3 sites}
% TODO: Sprachfeatures beschreiben
\section{Abstract Syntax Tree \textasciitilde 1 site}
\section{Controlflow Graph \textasciitilde 1 sites}
\section{Single Static Assignment Form \textasciitilde 1 sites}
\section{Constant Propagation \textasciitilde 1 sites}
\section{Consideration of return values}
\subsection{Intraprocedural}
\subsection{Interprocedural}
\section{Satisfiable Modulo Theory \textasciitilde 3 sites}
In recent decades the use of 
\subsection{Satisfiability Prover \textasciitilde 1 site}
\subsection{SMT-LIB \textasciitilde 1 sites}
SMT-LIB offers standardization to implementing SMT-Solvers by defining a common input- and output format, providing descriptions of background theories and more. % https://smtlib.cs.uiowa.edu/
By using the same input and output format SMT-Solvers may be exchanged at any time. 
This format is LISP-like and shares the same characteristics and being a functional programming language. 
For example mutations are not permitted and there is no possiblity of using traditional loops (which may be replaced by recursion).
Usually a SMT-LIB programm begins with declaring functions. Variables are also represented as functions. 
Afterwards the given Boolean Expression is formulated. Since there is no direct way to initialize a variable with a value, it has to be done by using the equality function. 
The last part contains an instruction for the SMT-Solver to check the satisfiability. 
Since solvers are implemented using backtracking it is also possible to gain the calculated values which fit the given expression, if it is satisfiable. 
SMT-Solvers may also implement the simplification of a given expression.