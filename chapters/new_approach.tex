%!TEX root = ../main.tex

% 
\chapter{Finding unreachable code using a SMT-Solver \textasciitilde 10 sites}
\label{cha:finding unreachable code using a smt-solver}
\emph{10 Sites}

The developed approach uses the \emph{Control-flow graph} as a basis like described in \ref{sec:building cfg}, but does not transform it into \emph{Single static Assignment} form.
During analysis the \emph{Control-flow graph} will be traversed and interpreted like the program will be executed.
After each interpretation state (e.g. \emph{Assignments}, \emph{Path Conditions}) will be added accordingly and must be taken into account.
Merging of state has to be handled correctly to make correct statements about the current value of a variable.
The state is represented in the form of a predicate, which may be checked by a system capable of determining satisfiability (e.g. an \emph{SMT-Solver}).
A satisfiable result will continue with the next block(s) and using the new state as a basis and flag this block as visitable.
An unsatisfiable result stops the computation at the current block and will not pursue to continue the current path.
Therefore this method is pessimistic.


% % Vorgangsweise beschreiben
% \section{Prerequisites}
% \subsection{Parsing the Abstract Syntax Tree \textasciitilde 1 site}
% % redundant?
% \subsection{Building the Control-flow Graph \textasciitilde 1 site}
% % redundant?
\section{Intermediate Representation}
\subsection{Parameters and Global Variables \textasciitilde 1 site}
\subsection{Function- and Procedure-calls \textasciitilde 2 sites}
\section{Analysis}
\subsection{Foundation \textasciitilde 1 site}
\subsection{State-management \textasciitilde 3 site}
\subsection{Creating combined SMTLib-Statements \textasciitilde 1 site}
\subsection{Interpret Result of SMT-Solver \textasciitilde 1 site}
